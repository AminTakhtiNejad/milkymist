\documentclass[a4paper,11pt]{article}
\usepackage{fullpage}
\usepackage[latin1]{inputenc}
\usepackage[T1]{fontenc}
\usepackage[normalem]{ulem}
\usepackage[english]{babel}
\usepackage{listings,babel}
\lstset{breaklines=true,basicstyle=\ttfamily}
\usepackage{graphicx}
\usepackage{moreverb}
\usepackage{url}

\title{Wishbone to FML caching bridge}
\author{S\'ebastien Bourdeauducq}
\date{\today}
\begin{document}
\setlength{\parindent}{0pt}
\setlength{\parskip}{5pt}
\maketitle{}
\section{Specifications}
This core gives access over a Wishbone link to a memory subsystem using the FML bus.

It can be used to connect the high-bandwidth parts of a system which use FML to a more traditional Wishbone system-on-chip base, which includes a CPU and low-speed peripherals.

To make efficient use of the burst-oriented FML bandwidth, the bridge implements a cache. This cache can cause coherency problems with other FML masters writing to the memory. To solve this issue, the bridge can be programmed to flush the cache.

\end{document}
