\documentclass[a4paper,11pt]{article}
\usepackage{fullpage}
\usepackage[latin1]{inputenc}
\usepackage[T1]{fontenc}
\usepackage[normalem]{ulem}
\usepackage[english]{babel}
\usepackage{listings,babel}
\lstset{breaklines=true,basicstyle=\ttfamily}
\usepackage{graphicx}
\usepackage{moreverb}
\usepackage{url}

\title{Wishbone bus arbiter and address decoder}
\author{S\'ebastien Bourdeauducq}
\date{October 2009}
\begin{document}
\setlength{\parindent}{0pt}
\setlength{\parskip}{5pt}
\maketitle{}
\section{Specifications}
This core allows up to several masters to communicate with up to several slaves on a shared Wishbone bus.

It takes care of bus arbitration and remapping of the slave base addresses. It is very simple and does not take care of priorities. Scheduling occurs when a master releases the bus, and then the next master which requested the bus takes ownership.

It is based on \verb!wb_conbus! from OpenCores.

\section{Using the core}
All parameters and ports should be self-explanatory. No special care should be taken.

\end{document}
