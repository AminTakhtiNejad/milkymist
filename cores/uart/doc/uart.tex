\documentclass[a4paper,11pt]{article}
\usepackage{fullpage}
\usepackage[latin1]{inputenc}
\usepackage[T1]{fontenc}
\usepackage[normalem]{ulem}
\usepackage[english]{babel}
\usepackage{listings,babel}
\lstset{breaklines=true,basicstyle=\ttfamily}
\usepackage{graphicx}
\usepackage{moreverb}
\usepackage{amsmath}
\usepackage{url}
\usepackage{tabularx}

\title{Simple UART}
\author{S\'ebastien Bourdeauducq}
\date{October 2009}
\begin{document}
\setlength{\parindent}{0pt}
\setlength{\parskip}{5pt}
\maketitle{}
\section{Specifications}
The UART is based on a very simple design from Das Labor. Its purpose is basically to provide a debug console.

The UART operates with 8 bits per character, no parity, and 1 stop bit. The default baudrate is configured during synthesis and can be modified at runtime using the divisor register.

The divisor is computed as follows :
\begin{equation*}
divisor = \frac{Clock frequency (Hz)}{16 \cdot Bitrate (bps)}
\end{equation*}

\section{Registers}

\begin{tabularx}{\textwidth}{|l|l|l|X|}
\hline
\bf{Offset} & \bf{Read/Write} & \bf{Default} & \bf{Description} \\
\hline
0x0 & RW & 0x00 & Control/status register.\\
& & & Bit 0 (read-only): RX data available.\\
& & & Bit 1 (read-only): RX error.\\
& & & Bit 2 (write-only): RX ack.\\
& & & Bit 3 (read-only): TX busy.\\
& & & Bit 4 (read-only): TX done.\\
& & & Bit 5 (write-only): TX ack. \\
\hline
0x4 & RW & 0x00 & Data register. Received bytes and bytes to transmit are read/written from/to this register. \\
\hline
0x8 & RW & for default bitrate & Divisor register (for bitrate selection). \\
\hline
\end{tabularx}\\

Upon reception of a byte, the register 0 should be written with the ``RX ack'' bit set to acknowledge the transmission (and clear the ``data available'' and ``error'' bits).

Upon transmission of a byte, the register 0 can be written with the ``TX ack'' bit set to clear the ``TX done'' bit. This is not mandatory if the transmission is not operating in an interrupt-driven mode (it is possible to busy-wait on the ``TX busy'' bit instead).

\section{Interrupts}
The core has two interrupts outputs. The ``RX'' interrupt is asserted whenever the ``RX data available'' or ``RX error'' bits are set. The ``TX'' interrupt is asserted whenever the ``TX done'' bit is set.


\section{Using the core}
Connect the CSR signals and the interrupts to the system bus and the interrupt controller. The \verb!uart_txd! and \verb!uart_rxd! signals should go to the FPGA pads. You must also provide the desired default baudrate and the system clock frequency in Hz using the parameters.

\end{document}
